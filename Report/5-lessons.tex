%!TEX root = report.tex

\chapter{Erfahrungen}
\label{cha:erfahrungen}

Die erste Lektion, die wir gelernt haben, ist, dass das Konzept der natürlichen Sprache Mehrdeutigkeiten mitbringen. Um diese Mehrdeutigkeiten zu verhindern, müssten wir Konstrukte wie Klammern oder erweiterte Interpunktion einführen, was jedoch mit diesem Konzept brechen würde. Als folge der Verwendung natürlicher Sprache wurde das gleichzeitige Erlernen und Implementieren der Sprache mit Hilfe von ANTLR zu einer Herausforderung, sodass viel Zeit auf diesen Teil verwendet werden musste und einige diskutierten Konzepte nicht umgesetzt werden konnten. Auch für die Implementierung konnte nur ein kleiner Anteil der Zeit investiert werden. Als Lehre daraus ziehen wir, dass wir in künftigen Neuentwicklungen auf eine formellere, aber dennoch leicht zu verstehende Sprache setzen würden. Statt nur auf die Sprache zu setzen, wäre es denkbar den Entwickler durch eine einfache, Baukastenartige Entwicklungsumgebung zu unterstützen.

Weiterhin haben wir versucht, möglichst viele Konzepte, die die Domäne mitbringt, in \dg umzusetzen. Im Laufe der Entwicklung haben wir festgestellt, dass eine strikte Umsetzung dieser Konzepte Nachteile mit sich bringt. Für Fortgeschrittene Nutzer sind weitergehende Konzepte, die nicht domäneninhärent sind wünschenswert. Hier ist an erste Stelle das Konzept der Vererbung im Sinne der Erweiterbarkeit von Spielen zu nennen.


Außerdem würden wir zunächst die benötigten Technologien lernen, sodass sich die Konzeption der Sprache und die Erlernung dieser nicht gegenseitig behindern. Auch eine iterative Entwicklung, bei der zunächst kleine Teile implementiert und diese anschließend erweitert werden ist wünschenswert. Die Wahl der zugrundeliegenden Technologien wie beispielsweise Parsergeneratoren sollte fundiert getroffen und ausführlich diskutiert werden, da verschiedene Technologien zu unterschiedlichen Zwecken entwickelt wurden.

Schließlich haben wir in der Umsetzung des Compilers einige Konzepte von \dg nicht direkt übernommen, was uns an manchen Punkten Schwierigkeiten bereitet hat, da bereits erdachte und diskutierte Lösungen umgeworfen werden mussten. Beispielsweise gibt es in \dg nur einen globalen Scope, der in Python jedoch unzulänglich umgesetzt ist. Auch eine bessere Wahl der Zielsprache könnte diskutiert werden. Es sollte eine Sprache gewählt werden, die der eigenen konzeptuell möglichst nah ist. In einer Neuimplementierung würden wir auf Java setzen, um das Framework als Sammlung von abstrakten Superklassen umzusetzen und die Spiele davon erben zu lassen. Eine Erweitarbarkeit von Spielen wäre ebenfalls einfach implementierbar. Denkbar wäre auch eine Implementierung, die nicht als Compiler sondern als Interpreter umgesetzt ist.