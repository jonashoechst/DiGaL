%!TEX root = report.tex

\chapter{Designentscheidungen}
\label{cha:designentscheidungen}

\section{Qualitäten der Sprache} % (fold)
\label{sec:qualitaten_der_sprache}


\subsection{Modularität} % (fold)
\label{sub:modularitat}

% subsection modularitat (end)

\subsection{Robustheit} % (fold)
\label{sub:robustheit}
Alle DiGaL Programme haben einen festen Rahmen, dessen freie Flächen durch Variablendefinitionen, Arithmetische Operationen und Kontrollflussanweisungen gefüllt werden. Im Gegensatz zu klassischen Programmiersprachen ist der Programmierer dadurch in seiner Mächtigkeit sehr eingeschränkt. Durch diese Einschränkung ist es ihm wiederum auch weniger leicht möglich, Fehler zu produzieren. Treten Programmierfehler im DiGaL Programm auf, sind diese schon durch die Syntax invalidiert und können nicht in ein Programm übersetzt werden. Als Beispiel sei hier die Iteration über eine Variable genannt, die in unserer Sprache nur als Iteration über eine Gruppe von Spielern oder Würfeln vorgesehen ist. 

Funktions- oder Klassendefinitionen sind in DiGaL nicht vorgesehen, der rein Iterative Ansatz reicht aus unserer Sicht aus, den Kontrollfluss von Würfelspielen zu beschreiben.
% subsection robustheit (end)

\subsection{Datenmodellierung} % (fold)
\label{sub:datenmodellierung}

% subsection datenmodellierung (end)

\subsection{Abstraktionsniveau} % (fold)
\label{sub:abstraktionsniveau}

% subsection abstraktionsniveau (end)

\subsection{Verständlichkeit} % (fold)
\label{sub:verstandlichkeit}

% subsection verstandlichkeit (end)

% section qualitaten_der_sprache (end)

\section{Alternativen im Sprachdesign} % (fold)
\label{sec:alternativen_im_sprachdesign}

% section alternativen_im_sprachdesign (end)

\section{Domänenbezug} % (fold)
\label{sec:domanenbezug}

% section domanenbezug (end)